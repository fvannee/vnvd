\section{Conclusies}
Wij hebben veel tijd en werk gestoken in deze compiler en het eindresultaat is dan ook te zien. We hebben een volwaardige compiler gebouwd voor onze eigen gedefinieerde taal. Tijdens het ontwerpen en implementeren zijn we veel moeilijkheden tegengekomen, maar hebben deze stuk voor stuk kunnen oplossen. Het resultaat is dan ook een mooie compiler met veel functies.

De opdracht is op verscheidene niveau's leerzaam geweest. Naast basiskennis over het opbouwen van een compiler hebben we ook veel inzichten vergaard over het, op conceptueel, ontwerpen van een taal en haar controleer en compileer systeem. De uiteindelijke taal is immers opgebouwd uit bewezen concepten, met daar onze eigen invulling aan. Er zijn immers niet veel object geori\"enteerde expressie talen.

Ook het .NET framework waarop de taal gebouwd is heeft ons goed geholpen. Door te kijken hoe constructies waren uitgewerkt in onze "moedertaal", C\#, konden we gebruik maken van de nuttige concepten. Daarnaast kregen we de mogelijkheid om te bedenken wat wij anders, en hopelijk beter, zouden doen en dit vervolgens ook uit te voeren.

Zeker het feit dat wij een expressietaal in een object geori\"enteerde mal hebben gegooid was in de eerste ontwikkel fase niet altijd even makkelijk. We kunnen nu echter succesvol concluderen dat het goed kan en enkele interessante taal mogelijkheden met zich meebrengt. Vooral het concept dat alles iets kan retourneren is een mooie insteek om mee te gaan programmeren. Dit geeft namelijk veel meer gevoel voor de achterliggende stack machine.

Al met al is VNVD het resultaat van veel werk en enthousiasme, waar we dan ook enige trots voor koesteren. Ons inziens is de taal zeker bruikbaar voor allerhande programmeer doeleinden. Of het nu grafische applicaties, servers, tools of libraries betreft, het is allemaal te maken en kost weinig moeite.

Indien we verder zouden willen gaan met VNVD zouden volgende stappen nog meer compatibiliteit met .NET standaarden, bootstrap en een IDE zijn. Waarschijnlijk zouden we ook enkele voor dit vak verplichte expressies, zoals de \textit{read()} en de \textit{write()}, verwijderen om meer in de geest, en abstractie, van de taal te blijven.

Er valt dus te concluderen dat VNVD een geslaagd en leuk project was, dat ook nog een uitermate bruikbaar product heeft opgeleverd.
